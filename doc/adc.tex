\section{ADC}
\label{sec:adc}
\textit{\hyperlink{schematic.4}{schematic}}

\subsection{Overview}
\label{sec:adc-overview}

The ADC is used to digitize signals amplified by the IF \hyperref[sec:ada4940-2]{ADA4940-2}
differential amplifiers before sending them to the \hyperref[sec:xc7a15t-ftg256]{FPGA} for
processing.

\subsection{LTC2292}
\label{sec:ltc2292}

\subsubsection{Description}
\label{sec:ltc2292-description}

The \href{http://www.analog.com/media/en/technical-documentation/data-sheets/229321fa.pdf}{LTC2292}
is a 40MHz, 12-bit differential input ADC\@. This sets the Nyquist frequency at $20 \si{MHz}$, well
above the several hundred $\si{kHz}$ of the signal frequencies. Oversampling allows us to relax
anti-aliasing filter requirements, which improves bandwidth and resolution.

\subsubsection{Pinout}
\label{sec:ltc2292-pinout}

\label{tab:ltc2292-pinout}
\begin{tabularx}{\textwidth}{>{\hsize=.25\hsize} X >{\hsize=.25\hsize} XX}
        \caption{All LTC2292 ADC pin connections in logical groupings.} \\
        \toprule
        LABELS & PIN \#s & DESCRIPTION \\
        \midrule

        VDD & 7, 10, 18, 63 & The power supply required is 3.0V and each pin requires its own 0.1$\mu$F
        capacitor. When designing the PCB, ensure that each capacitor is placed as close to its
        corresponding pin as possible to minimize the trace inductance. The 3.0V signal itself
        is generated by using an LP5907 LDO regulator that takes in a 3.6V signal.\\
        OVDD & 39, 42 & The power supply for the output that the ADC feeds to, which is the FPGA and takes
        a 3.3V power supply. So, we connect this to the 3.3V power rail and bypass it with two 0.1$\mu$F
        capacitors, one for each pin. Again, each should be placed adjacent
        to their corresponding pins (39 and 42). \\
        CLKA, CLKB, MUX & 8, 9, 21 & Feeds a 40MHz clock signal to the device and specifies that the
        digitized channel A and B data should be multiplexed and pass through both output buses A and
        B. We leave B unconnected, so only bus A matters. Fig.~\ref{fig:ltc2292-multiplex} shows the
        timing for this. \\
        DA0-DA11 & 43-48, 51-56 & The digitized output data that contains the input data from both
        channels A and B, multiplexed. This is fed into the FPGA for
        processing. \\
        OFA & 57 & This pin is driven high when overflow or underflow occurs. Otherwise it is kept low. We
        export this to the FPGA, although it is not currently used by the FPGA logic. \\
        DB0-DB11 & 26-30, 33-39 & The channel B data bus. Since we multiplex everything through the
        channel A data bus, this data bus is redundant and therefore left unconnected. \\
        OFB & 40 & Similar to OFA, but for channel B. This is exported to the FPGA, but is also unused. \\
        VCMA, VCMB & 61, 20 & A 1.5V signal that is used to set the common-mode voltage of the IF
        differential amplifiers, whose outputs are sent to this ADC. They are each bypassed to GND with a
        2.2$\mu$F capacitor, which should be placed directly
        next to their respective pins. \\
        GND, OGND & 17, 64, 65, 31, 50 & The ADC power ground and output power ground, respectively. These
        can all be routed to the same ground plane. \\
        NC & 24-25, 41-42 & No connect. \\
        SHDNA, SHDNB, OEA, OEB & 59, 22, 58, 23 & These are input pins that are connected to the FPGA. The
        FPGA logic can ground both SHDNA and OEA to allow channel A to operate normally or bring them both
        high to put channel A in sleep mode. Channel B works the same
        way. \\
        SENSEA, SENSEB & 62, 19 & These are connected to VDD, which specifies that the input voltage range
        of the differential signals for both channels A and B is 1.5V $\pm$ 1V. 1.5V is the common-mode
        voltage and the channels allow a 2V range
        around that. \\
        MODE & 60 & Connecting mode to VDD specifies the output format as 2s complement and turns of the
        clock duty stabilizer, which is unnecessary because the input clock has a 50\% duty
        cycle. \\
        REFHA, REFLA, REFLB, REFHB & 3-6, 11-14 & These are the high and low reference for channels A and
        B, respectively. Their connection is specified exactly by the datasheet. It is critical that the
        0.1$\mu$F
        capacitor is placed as close to the pins as possible. \\
        AINA+, AINA-, AINB-, AINB+ & 1-2, 15-16 & The positive and negative differential inputs for
        channels A and B. \textbf{\{STARTINCOMPLETE\}} These lines have a capacitor between the positive
        and negative analog inputs, as well as capacitors connecting each line to GND. The capacitor value
        between the lines of 0.1$\mu$F is different from the suggested value of 12pF and the capacitors
        connecting the lines to GND are not suggested from the datasheet (see page 18 of the
        datasheet). Additionally, the resistance value of 49.9$\Omega$ differs from the suggested
        resistance of 25$\Omega$. Lastly, the negative output of the differential amplifier for channel A
        is feeding into the
        positive input line \textbf{\{END INCOMPLETE\}}. \\

        \bottomrule
\end{tabularx}

\begin{figure}[h]
        \centering\includegraphics[width=0.75\textwidth]{data/LTC2292-multiplex.png}
        \caption{Multiplexed digital output bus timing for the LTC2292 ADC.}
        \label{fig:ltc2292-multiplex}
\end{figure}

\subsubsection{Component Selection}
\label{sec:ltc2292-component-selection}

The analog inputs employ anti-aliasing filters. The filters provide several benefits: (1) they
reduce aliasing by attenuating signals above the Nyquist frequency, (2) they limit wideband noise at
the input at the ADC input, which is important because the converter has a $575 \si{MHz}$ full-power
bandwidth, and (3) they isolate the \hyperref[sec:ada4940-2]{IF amplifier} from ADC noise at the
sampling frequency. Lastly, the capacitors act as a necessary charge source for the ADC input
capacitor. The two outer $100 \si{pF}$ capacitors (between the signal lines and ground) provide CMR
low-pass filtering with a cutoff frequency of $32 \si{MHz}$ (equation given in
Eq.~\ref{eq:anti-alias-cm-lp-filter}). The capacitor between the signal lines provides differential
low-pass filtering with a cutoff frequency of $16 \si{MHz}$. A
\href{https://e2e.ti.com/blogs_/archives/b/precisionhub/archive/2015/11/06/three-guidelines-for-designing-anti-aliasing-filters}{post
  from TI} explains this filtering. It's also important to keep the series resistor value low, since
the greater the resistance, the greater the Johnson noise.

\begin{equation}
        \label{eq:anti-alias-cm-lp-filter}
        f_{\text{c}} = \frac{1}{2 \pi R C}
\end{equation}

\subsubsection{PCB Layout}
\label{sec:ltc2292-pcb}



%%% Local Variables:
%%% mode: latex
%%% TeX-master: "fmcw-radar"
%%% End:

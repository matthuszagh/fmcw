\section{FPGA}
\label{sec:fpga}

\subsection{Overview}
\label{sec:fpga-overview}

The FPGA has two main roles. It performs digital signal processing on data sent from the ADC and
relays the processed data back to a host computer via an \hyperlink{sec:ft2232h}{FT2232H} which acts
as a FIFO-USB interface. Secondly, the FPGA is used to configure and control other components on the
board. In order to program the FPGA, we use a host computer with a JTAG interface via the FT2232H\@.

The FPGA stores and loads its configuration to/from an SPI flash memory device,
\hyperlink{sec:w25q32jv}{W25Q32JV}.

\subsection{XC7A15T-FTG256 Xilinx FPGA}
\label{sec:xc7a15t-ftg256}

\subsubsection{Description}
\label{sec:xc7a15t-ftg256-description}



\subsubsection{Linked Sections}
\label{sec:xc7a15t-ftg256-linked-sections}

\textit{\hyperref[sec:xc7a15t-ftg256-pinout]{pinout}}

\subsubsection{Documents}
\label{sec:xc7a15t-ftg256-documents}

\label{tab:xc7a15t-ftg256-documents}
\begin{tabularx}{\textwidth}{l X>{\raggedright\arraybackslash}X}
        \caption{Important documents.} \\
        \toprule
        \textbf{URL} & \textbf{Description} \\
        \midrule
        \href{https://www.xilinx.com/products/silicon-devices/fpga/artix-7.html?resultsTablePreSelect=documenttype:Data\%20Sheets\#documentation}{Artix-7
          FPGA family} & The main page that contains a collection of documents about the 7 series
        devices. \\
        \href{https://www.xilinx.com/support/documentation/user_guides/ug475_7Series_Pkg_Pinout.pdf}{Package
          Pinout} & The various packages that the device comes in, along with the associated pin
        definitions. \\
        \href{https://www.xilinx.com/support/documentation/data_sheets/ds180_7Series_Overview.pdf}{Overview}
        & Contains general device capability information, such as number of logic cells, DSP blocks,
        etc. \\
        \href{https://www.xilinx.com/support/documentation/user_guides/ug483_7Series_PCB.pdf}{PCB
          Design Guide} &
        Contains information about bypassing and other design considerations. \\
        \href{https://www.xilinx.com/support/documentation/user_guides/ug480_7Series_XADC.pdf}{XADC}
        & Contains information about the FPGA's internal ADC\@. \\
        \href{https://www.xilinx.com/support/documentation/user_guides/ug470_7Series_Config.pdf}{Configuration
          Guide} & Describes how to configure the FPGA with a bitstream. \\
        \bottomrule
\end{tabularx}

\subsubsection{Power Supply Requirements}
\label{sec:xc7a15t-ftg256-power}

The FPGA requires that the VCCINT and VCCBRAM voltages be supplied first, followed by VCCAUX and
then by VCCO\@. Additionally, all this must be done between $0.2$ and $50 \si{ms}$. The required
voltage levels are displayed in Table~\ref{tab:xc7a15t-ftg256-power}.

\label{tab:xc7a15t-ftg256-power}
\begin{tabularx}{\textwidth}{l c c c}
        \caption{XC7A15T-FTG256 power supply requirements.} \\
        \toprule
        Voltage Pin & Min & Typ & Max \\
        \midrule
        V\textsubscript{CCINT} & 0.95 & 1.00 & 1.05 \\
        V\textsubscript{CCBRAM} & 0.95 & 1.00 & 1.05 \\
        V\textsubscript{CCAUX} & 1.71 & 1.80 & 1.89 \\
        V\textsubscript{CCO} & 1.14 & - & 3.465 \\
        \bottomrule
\end{tabularx}

\subsubsection{Bypassing}
\label{sec:xc7a15t-ftg256-bypassing}

The PCB Design Guide above specifies fairly stringent requirements for bypassing and should be
adhered to.

\subsubsection{PCB Layout}
\label{sec:xc7a15t-ftg256-pcb}



\subsection{W25Q32JV Flash Memory}
\label{sec:w25q32jv}

The CCLK\_0 pin is exported from the FPGA to the flash device to drive its operation and coordinates
writes and reads to and from the device. DO is used by the FPGA to perform SPI reads from the flash
memory device. It is connected to J14 of the FPGA\@. DI is connected to J13 of the FPGA and is used
by the FPGA to send data to the flash memory device. CS is active low and is used by the FPGA to
signal data transmission is about to occur. It is connected to L12 on the FPGA\@. WP (write protect)
and HOLD are both active low pins and are used to prevent the status configuration registers from
being written to and can pause the device when multiple devices share the same SPI signal,
respectively They are unused and thus connected to the 3.3V power supply. The device is powered with
3.3V (it supports a range of 2.7V to 3.6V). Table~\ref{tab:fpga-pins} contains a list of all FPGA
pins and their connections.

%%% Local Variables:
%%% mode: latex
%%% TeX-master: "fmcw-radar"
%%% End:
